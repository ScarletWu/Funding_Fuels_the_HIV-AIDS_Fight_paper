% Options for packages loaded elsewhere
\PassOptionsToPackage{unicode}{hyperref}
\PassOptionsToPackage{hyphens}{url}
\PassOptionsToPackage{dvipsnames,svgnames,x11names}{xcolor}
%
\documentclass[
  letterpaper,
  DIV=11,
  numbers=noendperiod]{scrartcl}

\usepackage{amsmath,amssymb}
\usepackage{iftex}
\ifPDFTeX
  \usepackage[T1]{fontenc}
  \usepackage[utf8]{inputenc}
  \usepackage{textcomp} % provide euro and other symbols
\else % if luatex or xetex
  \usepackage{unicode-math}
  \defaultfontfeatures{Scale=MatchLowercase}
  \defaultfontfeatures[\rmfamily]{Ligatures=TeX,Scale=1}
\fi
\usepackage{lmodern}
\ifPDFTeX\else  
    % xetex/luatex font selection
\fi
% Use upquote if available, for straight quotes in verbatim environments
\IfFileExists{upquote.sty}{\usepackage{upquote}}{}
\IfFileExists{microtype.sty}{% use microtype if available
  \usepackage[]{microtype}
  \UseMicrotypeSet[protrusion]{basicmath} % disable protrusion for tt fonts
}{}
\makeatletter
\@ifundefined{KOMAClassName}{% if non-KOMA class
  \IfFileExists{parskip.sty}{%
    \usepackage{parskip}
  }{% else
    \setlength{\parindent}{0pt}
    \setlength{\parskip}{6pt plus 2pt minus 1pt}}
}{% if KOMA class
  \KOMAoptions{parskip=half}}
\makeatother
\usepackage{xcolor}
\setlength{\emergencystretch}{3em} % prevent overfull lines
\setcounter{secnumdepth}{5}
% Make \paragraph and \subparagraph free-standing
\ifx\paragraph\undefined\else
  \let\oldparagraph\paragraph
  \renewcommand{\paragraph}[1]{\oldparagraph{#1}\mbox{}}
\fi
\ifx\subparagraph\undefined\else
  \let\oldsubparagraph\subparagraph
  \renewcommand{\subparagraph}[1]{\oldsubparagraph{#1}\mbox{}}
\fi


\providecommand{\tightlist}{%
  \setlength{\itemsep}{0pt}\setlength{\parskip}{0pt}}\usepackage{longtable,booktabs,array}
\usepackage{calc} % for calculating minipage widths
% Correct order of tables after \paragraph or \subparagraph
\usepackage{etoolbox}
\makeatletter
\patchcmd\longtable{\par}{\if@noskipsec\mbox{}\fi\par}{}{}
\makeatother
% Allow footnotes in longtable head/foot
\IfFileExists{footnotehyper.sty}{\usepackage{footnotehyper}}{\usepackage{footnote}}
\makesavenoteenv{longtable}
\usepackage{graphicx}
\makeatletter
\def\maxwidth{\ifdim\Gin@nat@width>\linewidth\linewidth\else\Gin@nat@width\fi}
\def\maxheight{\ifdim\Gin@nat@height>\textheight\textheight\else\Gin@nat@height\fi}
\makeatother
% Scale images if necessary, so that they will not overflow the page
% margins by default, and it is still possible to overwrite the defaults
% using explicit options in \includegraphics[width, height, ...]{}
\setkeys{Gin}{width=\maxwidth,height=\maxheight,keepaspectratio}
% Set default figure placement to htbp
\makeatletter
\def\fps@figure{htbp}
\makeatother
% definitions for citeproc citations
\NewDocumentCommand\citeproctext{}{}
\NewDocumentCommand\citeproc{mm}{%
  \begingroup\def\citeproctext{#2}\cite{#1}\endgroup}
\makeatletter
 % allow citations to break across lines
 \let\@cite@ofmt\@firstofone
 % avoid brackets around text for \cite:
 \def\@biblabel#1{}
 \def\@cite#1#2{{#1\if@tempswa , #2\fi}}
\makeatother
\newlength{\cslhangindent}
\setlength{\cslhangindent}{1.5em}
\newlength{\csllabelwidth}
\setlength{\csllabelwidth}{3em}
\newenvironment{CSLReferences}[2] % #1 hanging-indent, #2 entry-spacing
 {\begin{list}{}{%
  \setlength{\itemindent}{0pt}
  \setlength{\leftmargin}{0pt}
  \setlength{\parsep}{0pt}
  % turn on hanging indent if param 1 is 1
  \ifodd #1
   \setlength{\leftmargin}{\cslhangindent}
   \setlength{\itemindent}{-1\cslhangindent}
  \fi
  % set entry spacing
  \setlength{\itemsep}{#2\baselineskip}}}
 {\end{list}}
\usepackage{calc}
\newcommand{\CSLBlock}[1]{\hfill\break\parbox[t]{\linewidth}{\strut\ignorespaces#1\strut}}
\newcommand{\CSLLeftMargin}[1]{\parbox[t]{\csllabelwidth}{\strut#1\strut}}
\newcommand{\CSLRightInline}[1]{\parbox[t]{\linewidth - \csllabelwidth}{\strut#1\strut}}
\newcommand{\CSLIndent}[1]{\hspace{\cslhangindent}#1}

\KOMAoption{captions}{tableheading}
\makeatletter
\@ifpackageloaded{caption}{}{\usepackage{caption}}
\AtBeginDocument{%
\ifdefined\contentsname
  \renewcommand*\contentsname{Table of contents}
\else
  \newcommand\contentsname{Table of contents}
\fi
\ifdefined\listfigurename
  \renewcommand*\listfigurename{List of Figures}
\else
  \newcommand\listfigurename{List of Figures}
\fi
\ifdefined\listtablename
  \renewcommand*\listtablename{List of Tables}
\else
  \newcommand\listtablename{List of Tables}
\fi
\ifdefined\figurename
  \renewcommand*\figurename{Figure}
\else
  \newcommand\figurename{Figure}
\fi
\ifdefined\tablename
  \renewcommand*\tablename{Table}
\else
  \newcommand\tablename{Table}
\fi
}
\@ifpackageloaded{float}{}{\usepackage{float}}
\floatstyle{ruled}
\@ifundefined{c@chapter}{\newfloat{codelisting}{h}{lop}}{\newfloat{codelisting}{h}{lop}[chapter]}
\floatname{codelisting}{Listing}
\newcommand*\listoflistings{\listof{codelisting}{List of Listings}}
\makeatother
\makeatletter
\makeatother
\makeatletter
\@ifpackageloaded{caption}{}{\usepackage{caption}}
\@ifpackageloaded{subcaption}{}{\usepackage{subcaption}}
\makeatother
\ifLuaTeX
  \usepackage{selnolig}  % disable illegal ligatures
\fi
\usepackage{bookmark}

\IfFileExists{xurl.sty}{\usepackage{xurl}}{} % add URL line breaks if available
\urlstyle{same} % disable monospaced font for URLs
\hypersetup{
  pdftitle={Tracing Hidden Struggles of Rural Indian Women during Pandemic},
  pdfauthor={Ruoxian Wu},
  colorlinks=true,
  linkcolor={blue},
  filecolor={Maroon},
  citecolor={Blue},
  urlcolor={Blue},
  pdfcreator={LaTeX via pandoc}}

\title{Tracing Hidden Struggles of Rural Indian Women during
Pandemic\thanks{Code and data are available at:
https://github.com/ScarletWu/Tracing\_Hidden\_Struggles\_of\_Rural\_Indian\_Women\_during\_Pandemic.git.
Replication on Social Science Reproduction platform is available at:
https://www.socialsciencereproduction.org/reproductions/1783/}}
\author{Ruoxian Wu}
\date{April 16, 2024}

\begin{document}
\maketitle

\renewcommand*\contentsname{Table of contents}
{
\hypersetup{linkcolor=}
\setcounter{tocdepth}{3}
\tableofcontents
}
\section{Introduction}\label{introduction}

Public health funding significantly impacts public health outcomes,
particularly during global health emergencies like the COVID-19
pandemic. In the study ``Women's well-being during a pandemic and its
containment'' published in the Journal of Development Economics (2022),
the authors explore the dual crisis of disease and the containment
policies designed to mitigate its spread, focusing specifically on their
effects on women in lower-income countries. This paper critically
analyzes the methodologies and findings of the original study, aiming to
understand the nuanced impacts of such policies on various aspects of
women's well-being, including mental health and food security.

In my analysis, I replicate the original study's research using publicly
available data, extending the investigation to assess how containment
measures affect the broader socio-economic outcomes beyond what was
initially reported. Utilizing methodologies such as
difference-in-differences and regression discontinuity designs, this
paper evaluates the sensitivity of the findings to different analytical
approaches. By doing so, it seeks to validate the original results while
stimulating further discussion on interpreting data to assess public
health policies' impacts.

This critical analysis endeavors to unravel the complex relationships
between public health funding, policy effectiveness, and their
socio-economic ramifications. Through a detailed reexamination of the
original study's data and methodology, this paper contributes to a more
comprehensive understanding of how public health initiatives can be
optimized to better serve vulnerable populations during crises. My
analysis keeps a keen eye on the societal, economic, and systemic
factors that shape public health outcomes and the operational dynamics
of public funding within these contexts.

\section{Data}\label{data}

My reproduction used the programming language R (R Core Team 2022), the
analysis used the following packages: Haven(Wickham, Miller, and Smith
2023), Dplyr (Wickham et al. 2023), Ggplot2 (Wickham 2016), Readr
(Wickham, Hester, and Bryan 2024), Here (Müller 2020), Janitor (Firke
2023), KableExtra (Zhu 2024), Knitr (Xie 2014), Tidyverse
(\textbf{rTidyverse?}).

\subsection{Source}\label{source}

This critical analysis utilizes replication data associated with the
article ``Women's well-being during a pandemic and its containment''
from the Journal of Development Economics. This data, along with
associated code, was made accessible by the authors to facilitate the
replication of key findings such as statistical models and graphical
representations. By enabling the reproduction of the authors' analyses,
this data contributes to the transparency and credibility of the study's
conclusions. The replication package can be find and downloaded after
requesting access from .

\subsection{Variables}\label{variables}

The data for this study includes both individual-level and
regional-level variables from six states in rural India.
Individual-level variables encompass demographic details (age, gender,
household head status), economic factors (employment status, income
levels), and health-related outcomes (mental health indicators,
nutrition data). Regional-level variables cover containment measures,
healthcare access, and socio-economic indicators such as the prevalence
of COVID-19, public health infrastructure, and local economic
conditions.

Regarding data collection, the authors conducted a large phone survey in
August 2020, targeting households that were first interviewed in the
fall of 2019, thereby providing a pre-pandemic baseline. This
longitudinal approach allowed the researchers to examine changes over
time attributed to the pandemic and containment policies. The survey
data were supplemented with regional health statistics and COVID-19 case
data obtained from official public health sources.

The data from the phone survey included detailed questions on mental
health using validated psychological scales (PHQ9 and GAD7) and food
security questions adapted from national health surveys. This allowed
the researchers to construct indices of mental health and nutritional
status, crucial for evaluating the impact of containment policies on
women's well-being.

Subsequent to data collection, the data were organized and analyzed
using statistical software, with the authors employing advanced
econometric techniques such as difference-in-differences and regression
discontinuity designs to assess the impact of varying levels of
containment. This rigorous analytical approach helps to isolate the
effects of public health interventions from other confounding factors.

\section{Results}\label{results}

\section{Discussion}\label{discussion}

In progress

\section*{Reference}\label{reference}
\addcontentsline{toc}{section}{Reference}

\phantomsection\label{refs}
\begin{CSLReferences}{1}{0}
\bibitem[\citeproctext]{ref-rJanitor}
Firke, Sam. 2023. \emph{Janitor: Simple Tools for Examining and Cleaning
Dirty Data}. \url{https://github.com/sfirke/janitor}.

\bibitem[\citeproctext]{ref-rHere}
Müller, Kirill. 2020. \emph{Here: A Simpler Way to Find Your Files}.
\url{https://here.r-lib.org/}.

\bibitem[\citeproctext]{ref-r}
R Core Team. 2022. \emph{R: A Language and Environment for Statistical
Computing}. Vienna, Austria: R Foundation for Statistical Computing.
\url{https://www.R-project.org/}.

\bibitem[\citeproctext]{ref-rGgplot2}
Wickham, Hadley. 2016. \emph{Ggplot2: Elegant Graphics for Data
Analysis}. Springer-Verlag New York.
\url{https://ggplot2.tidyverse.org}.

\bibitem[\citeproctext]{ref-rDplyr}
Wickham, Hadley, Romain François, Lionel Henry, Kirill Müller, and Davis
Vaughan. 2023. \emph{Dplyr: A Grammar of Data Manipulation}.
\url{https://dplyr.tidyverse.org}.

\bibitem[\citeproctext]{ref-rReadr}
Wickham, Hadley, Jim Hester, and Jennifer Bryan. 2024. \emph{Readr: Read
Rectangular Text Data}. \url{https://readr.tidyverse.org}.

\bibitem[\citeproctext]{ref-rHaven}
Wickham, Hadley, Evan Miller, and Danny Smith. 2023. \emph{Haven: Import
and Export 'SPSS', 'Stata' and 'SAS' Files}.
\url{https://haven.tidyverse.org}.

\bibitem[\citeproctext]{ref-rKnitr}
Xie, Yihui. 2014. {``Knitr: A Comprehensive Tool for Reproducible
Research in {R}.''} In \emph{Implementing Reproducible Computational
Research}, edited by Victoria Stodden, Friedrich Leisch, and Roger D.
Peng. Chapman; Hall/CRC.
\url{http://www.crcpress.com/product/isbn/9781466561595}.

\bibitem[\citeproctext]{ref-rKableExtra}
Zhu, Hao. 2024. \emph{kableExtra: Construct Complex Table with 'Kable'
and Pipe Syntax}. \url{http://haozhu233.github.io/kableExtra/}.

\end{CSLReferences}



\end{document}
